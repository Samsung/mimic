%
%
\documentclass[]{article}

\usepackage[utf8]{inputenc}
\usepackage[T1]{fontenc}
%\usepackage{lmodern}
%\usepackage{textcomp}
%\usepackage[scaled=0.88]{beramono}
\usepackage{graphicx}

\usepackage{ucs}
\usepackage{longtable}

% At least 80% of every float page must be taken up by
% floats; there will be no page with more than 20% white space.
\def\topfraction{.95}
\def\dbltopfraction{\topfraction}
\def\floatpagefraction{\topfraction}     % default .5
\def\dblfloatpagefraction{\topfraction}  % default .5
\def\textfraction{.05}

\usepackage{amsthm}

\usepackage{listings}
\lstset{
    language=Java,
    basicstyle=\ttfamily\mdseries,
    identifierstyle=,
    stringstyle=\color{gray},
    numbers=left,
    numbersep=5pt,
    inputencoding=utf8x,
    xleftmargin=8pt,
    xrightmargin=8pt,
    keywordstyle=[1]\bfseries,
    keywordstyle=[2]\bfseries,
    keywordstyle=[3]\bfseries,
    keywordstyle=[4]\bfseries,
    numberstyle=\tiny,
    stepnumber=1,
    breaklines=true,
    frame=lines,
    showstringspaces=false,
    tabsize=2,
    commentstyle=\color{gray},
    captionpos=b
}
\newcommand{\code}[1]{\lstinline{#1}}

\newcommand{\bigO}[1]{\mathcal O\left(#1\right)} % big-o notation

\usepackage{fullpage}

\def\sectionautorefname{Section}
\def\subsectionautorefname{Section}
\def\subsubsectionautorefname{Section}
\def\paragraphautorefname{Section}
\def\definitionautorefname{Definition}
\def\subsubsubsection{\paragraph}
\setcounter{secnumdepth}{4}


% macros for bnf grammers
\newenvironment{bnfgrammar}{\begin{center}\renewcommand{\arraystretch}{1.5}\begin{tabular}{rcl}}{\end{tabular}\end{center}}
\newcommand{\nonterminal}[1]{$\langle \textit{#1} \rangle$}
\newcommand{\production}[2]{#1 &$::=$& #2\\}
\newcommand{\literal}[1]{`$\text{\code{#1}}$'}
\newcommand{\alt}{$~\!\!\mid~$}
\newcommand{\altline}[1]{&$\mid$&#1\\}

\usepackage{mathtools}
\usepackage{amssymb,amsfonts,amsmath}
\usepackage[svgnames]{xcolor}
\usepackage[colorlinks=true,pdfborder={0 0 0},citecolor=DarkGreen,linkcolor=DarkBlue,urlcolor=DarkBlue]{hyperref}

\usepackage{amsthm}
% \usepackage{thmtools}
%\theoremstyle{definition}
\newtheorem{definition}{Definition}[section]

\usepackage{framed}
\newenvironment{workinprogress}
{\color{gray} \begin{leftbar}}
{\end{leftbar}}

\newenvironment{new}{\begin{framed}}{\end{framed}}

\usepackage{booktabs}

\newcommand{\todo}[1]{{\textcolor{DarkRed}{ [\textbf{TODO}: #1]}}}

\parindent 0em
\setlength{\parskip}{.5em}

\title{Synthesis of Library Models via Dynamic Probing}
\author{
  Stefan Heule\\\url{sheule@cs.stanford.edu}
  \and
  Manu Sridharan\\\url{m.sridharan@samsung.com}
}

\begin{document}

\maketitle



\section*{Motivation}

For any program analysis, static or dynamic, a key problem is handling of code
that, for one reason or another, cannot be analyzed.  Some code is either
unavailable or implemented in a difficult-to-analyze native language, e.g.,
implementations of the standard ECMAScript and DOM libraries in the browser.
Other code is simply difficult to analyze due to its size and complexity, e.g.,
frameworks like jQuery or even the Java standard library.  In realistic systems,
such code is often handled via hand-written models that expose its behavior to
analyses.  However, as the amount of such unanalyzable code grows, manual
writing of models becomes intractable. Models must often be customized to
individual analyses in order to be useful, making it very difficult to re-use
models written for one analysis when a new analysis is developed.  A system to
automatically generate library models in a manner customized to the desired
client analysis could be of great use in making program analyses more easily
applicable to real systems.  \todo{clarify usefulness for scenario of performing
  dynamic analysis where code is available.  Could provide speedups?}


\section*{Our approach}

We propose a technique for synthesizing library models via dynamic probing and
randomized search.  To learn the behaviors of a function without needing to
instrument its code, we repeatedly invoke the function with proxied objects as
arguments.  A proxied object allows for various low-level operations on the
object, such as field accesses, to be intercepted and customized.  Leveraging
this proxy functionality, we can dynamically probe the execution of library
functions and observe details of its behavior, even if the function’s code is
unavailable.  Proxy functionality is supported by a number of dynamic languages,
such as JavaScript, Python, and Ruby.

Executing a function with proxied arguments yields a trace of certain types of
operations it performed.    \todo{Add example} Given this core trace generation
functionality, our model synthesis proceeds as follows.   We are given as inputs
to the synthesizer a function $f$ to model and an initial set of inputs $i$
to $f$.  We first generate several other inputs to $f$ by an iterative process:
We use the current set of inputs to probe how the function uses these arguments,
and in particular which parts of the arguments get accessed by the function.
We create new inputs that differ in those parts, and use some further heuristics
to create interesting inputs (e.g., passing the empty array where an array is
expected, or passing both \code{true} and \code{false} for boolean arguments).
We then run $f$ on
all inputs from the generated set $I$ to get a corresponding set of traces $T$.

Given $T$, we generate an initial model m for f; for now, assume a model is simply
a JavaScript program with straight-line code.  By running m on each input in I,
we get a set of traces T'.  We can then compare the corresponding traces in T
and T' to compute a fitness score for m, corresponding to how well it models f.
Critically, this comparison of traces incorporates a predicate provided by the
desired analysis, so that trace aspects irrelevant to the analysis can be
ignored.  \todo{example} Given the model-scoring functionality, we perform a
hill-climbing search by randomly mutating m, keeping mutations that improve its
fitness score.  Mutations include \todo{give examples} When we have generated a
model m that obtains a ``perfect'' fitness score, i.e., the traces for that
model perfectly match the original function (incorporating the needs of the
analysis), we terminate.

We can extend the above technique to also generate models with conditionals, as
follows.  \todo{describe conditional handling}


\section*{Other Concerns}

\begin{itemize}
  \item hard-to-execute functions which require a bunch of other setup code to
  run first
  \item handling DOM methods, where it's hard to undo changes.  In general,
  handling methods that have side effects on the network, file system, etc.
  \item what happens when we miss a case and model is insufficient for dynamic
  analysis.  Simple answer: we just ask the analysis writer to provide a model in
  those cases.  (Could also ask them to provide a template of a model.)  Or, we
  could take the input that is not handled and re-run the synthesis technique with
  that input added.
\end{itemize}











% \bibliographystyle{alpha}
% \bibliography{bib.bib}

\end{document}

